
\documentclass[11pt,twoside,openright,largecrownvopaper]
{memoir}
\usepackage{geometry}
\setulmarginsandblock{2.38cm}{2.38cm}{*}
\setlrmarginsandblock{2.29cm}{1.52cm}{*}
\checkandfixthelayout
\clearmark{chapter}
\usepackage[pagestyles]{titlesec}

\newcommand\invisiblesection[1]{%
  \refstepcounter{section}%
  \addcontentsline{toc}{section}{\protect\numberline{\thesection}#1}%
  \sectionmark{#1}}

\renewcommand{\sectionmark}[1]{\markright{#1}}

\makepagestyle{mystyle}
\makeevenhead{mystyle}{\thepage\ \MakeUppercase\rightmark}{}{}
\makeoddhead{mystyle}{}{}{\MakeUppercase\rightmark\ \thepage}

\newcommand\anfang[2]{
\newpage
\invisiblesection{#1}
#2\ldots
}

\pagestyle{mystyle}
\chapterstyle{bringhurst}

\begin{document}

\chapter{100 Buchanf\"ange}
\thispagestyle{empty}
Eine sinnunige Zusammenstellung von\\\\
Simon Clemens Goslar\\
Johannes Bernhard Goslar\\
\\
Wieso endige Notizplätze, wenn innendrin mehr Leerraum verkaufbar ist.

\anfang{Irsebel}{Irsebel pfefferte nach}
\setcounter{page}{0}

\anfang{Breaker}{4537. 4327. 8476.}

\anfang{Simon}{Trank. Zauber. Besaufen.}

\end{document}
